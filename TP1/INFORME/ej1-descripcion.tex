 La comunicacion es el progreso! decididos a entrar de lleno en la nueva era el paıs decidio conectar
telegraficamente todas las estaciones del moderno sistema ferreo que recorre el paıs en abanico con origen
en la capita (el kilometro 0). Por lo escaso del presupuesto, se ha decidido ofrecer cierta cantidad de
kilometros de cable a cada ramal. Pero para maximizar el impacto en epocas electorales se busca lograr
conectar la mayor cantidad de ciudades con los metros asignados (sin hacer cortes en el cable)\\\\
Resolver cuantas ciudades se pueden conectar para cada ramal en O(n) , con n la cantidad de estaciones
en cada ramal, y justificar por que el procedimiento desarrollado resuelve efectivamente el problema.\\\\
Entrada \textbf{Tp1Ej1.in}\\\\
Cada ramal ocupa dos lıneas, la primera contiene un entero con los kilometros de cable dedicados al
ramal y la segunda los kilometrajes de las estaciones en el ramal sin cosiderar el 0.\\\\
Salida \textbf{Tp1Ej1.out}\\\\
Para cada ramal de entrada, se debe indicar una lınea con la cantidad de ciudades conectables encontradas.\\